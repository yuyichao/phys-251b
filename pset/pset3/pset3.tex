\documentclass[10pt,fleqn]{article}
\newcommand{\name}[1]{\def\psettitlename{#1}}
\newcommand{\course}[1]{\def\psettitlecourse{#1}}
\newcommand{\rsection}[1]{\def\psettitlersection{#1}}
\newcommand{\psetnum}[1]{\def\psettitlepsetnum{#1}}
% \usepackage[journal=rsc]{chemstyle}
% \usepackage{mhchem}
\usepackage{amsmath}
\usepackage{amssymb}
\usepackage{amsfonts}
\usepackage{esint}
\usepackage{bbm}
\usepackage{amscd}
\usepackage{picinpar}
\usepackage[pdftex]{graphicx}
\usepackage{indentfirst}
\usepackage{wrapfig}
\usepackage{units}
\usepackage{textcomp}
\usepackage[utf8x]{inputenc}
\usepackage{feyn}
\usepackage{feynmp}
\DeclareGraphicsRule{*}{mps}{*}{}
\newcommand{\ud}{\mathrm{d}}
\newcommand{\ue}{\mathrm{e}}
\newcommand{\ui}{\mathrm{i}}
\newcommand{\res}{\mathrm{Res}}
\newcommand{\Tr}{\mathrm{Tr}}
\newcommand{\dsum}{\displaystyle\sum}
\newcommand{\dprod}{\displaystyle\prod}
\newcommand{\dlim}{\displaystyle\lim}
\newcommand{\dint}{\displaystyle\int}
\newcommand{\fsno}[1]{{\!\not\!{#1}}}
\newcommand{\eqar}[1]
{
  \begin{align*}
    #1
  \end{align*}
}
\newcommand{\texp}[2]{\ensuremath{{#1}\times10^{#2}}}
\newcommand{\dexp}[2]{\ensuremath{{#1}\cdot10^{#2}}}
\newcommand{\eval}[2]{{\left.{#1}\right|_{#2}}}
\newcommand{\paren}[1]{{\left({#1}\right)}}
\newcommand{\lparen}[1]{{\left({#1}\right.}}
\newcommand{\rparen}[1]{{\left.{#1}\right)}}
\newcommand{\abs}[1]{{\left|{#1}\right|}}
\newcommand{\sqr}[1]{{\left[{#1}\right]}}
\newcommand{\crly}[1]{{\left\{{#1}\right\}}}
\newcommand{\angl}[1]{{\left\langle{#1}\right\rangle}}
\newcommand{\tpdiff}[4][{}]{{\paren{\frac{\partial^{#1} {#2}}{\partial {#3}{}^{#1}}}_{#4}}}
\newcommand{\tpsdiff}[4][{}]{{\paren{\frac{\partial^{#1}}{\partial {#3}{}^{#1}}{#2}}_{#4}}}
\newcommand{\pdiff}[3][{}]{{\frac{\partial^{#1} {#2}}{\partial {#3}{}^{#1}}}}
\newcommand{\diff}[3][{}]{{\frac{\ud^{#1} {#2}}{\ud {#3}{}^{#1}}}}
\newcommand{\psdiff}[3][{}]{{\frac{\partial^{#1}}{\partial {#3}{}^{#1}} {#2}}}
\newcommand{\sdiff}[3][{}]{{\frac{\ud^{#1}}{\ud {#3}{}^{#1}} {#2}}}
\newcommand{\tpddiff}[4][{}]{{\left(\dfrac{\partial^{#1} {#2}}{\partial {#3}{}^{#1}}\right)_{#4}}}
\newcommand{\tpsddiff}[4][{}]{{\paren{\dfrac{\partial^{#1}}{\partial {#3}{}^{#1}}{#2}}_{#4}}}
\newcommand{\pddiff}[3][{}]{{\dfrac{\partial^{#1} {#2}}{\partial {#3}{}^{#1}}}}
\newcommand{\ddiff}[3][{}]{{\dfrac{\ud^{#1} {#2}}{\ud {#3}{}^{#1}}}}
\newcommand{\psddiff}[3][{}]{{\frac{\partial^{#1}}{\partial{}^{#1} {#3}} {#2}}}
\newcommand{\sddiff}[3][{}]{{\frac{\ud^{#1}}{\ud {#3}{}^{#1}} {#2}}}
\usepackage{fancyhdr}
\usepackage{multirow}
\usepackage{fontenc}
% \usepackage{tipa}
\usepackage{ulem}
\usepackage{color}
\usepackage{cancel}
\newcommand{\hcancel}[2][black]{\setbox0=\hbox{#2}%
  \rlap{\raisebox{.45\ht0}{\textcolor{#1}{\rule{\wd0}{1pt}}}}#2}
\pagestyle{fancy}
\setlength{\headheight}{67pt}
\fancyhead{}
\fancyfoot{}
\fancyfoot[C]{\thepage}
\fancyhead[R]
{
  \psettitlename \\
  \psettitlecourse{} Problem Set \psettitlepsetnum \\
  \ifx\psettitlersection\empty
  \else
  Recitation Section \psettitlersection
  \fi
}
\renewcommand{\footruleskip}{0pt}
\renewcommand{\headrulewidth}{0.4pt}
\renewcommand{\footrulewidth}{0pt}
\addtolength{\hoffset}{-1.3cm}
\addtolength{\voffset}{-2cm}
\addtolength{\textwidth}{3cm}
\addtolength{\textheight}{2.5cm}
\renewcommand{\footskip}{10pt}
\setlength{\headwidth}{\textwidth}
\setlength{\headsep}{20pt}
\setlength{\marginparwidth}{0pt}
\parindent=0pt
\psetnum{3}
\course{Physics 251b}
\rsection{1}
\name{Yichao Yu}
\renewcommand{\thesection}{\arabic{section}.}
\renewcommand{\thesubsection}{(\alph{subsection})}
\renewcommand{\thesubsubsection}{\roman{subsubsection}.}

\begin{document}
\section{}
\subsection{}
Since $\ui\paren{\lambda a^\dagger-\lambda^*a}$ is Hermitian,
$S_\lambda\equiv\exp\paren{\lambda a^\dagger-\lambda^*a}$ is unitary
and $|\lambda\rangle\equiv S_\lambda|0\rangle$ is normalized.

Since $[a, a^\dagger]=1$ commutes with both $a$ and $a^\dagger$
\eqar{
  |\lambda\rangle=&\exp\paren{\lambda a^\dagger-\lambda^*a}|0\rangle\\
  =&\exp\paren{\lambda a^\dagger}\exp\paren{-\lambda^*a}\exp\paren{-\frac12\sqr{\lambda a^\dagger,-\lambda^*a}}|0\rangle\\
  =&\exp\paren{\lambda a^\dagger}\exp\paren{-\lambda^*a}\exp\paren{-\frac{\abs{\lambda}^2}2}|0\rangle\\
  =&\exp\paren{-\frac{\abs{\lambda}^2}2}\exp\paren{\lambda a^\dagger}|0\rangle
}
\eqar{
  a|\lambda\rangle=&\exp\paren{-\frac{\abs{\lambda}^2}2}a\exp\paren{\lambda a^\dagger}|0\rangle\\
  =&\exp\paren{-\frac{\abs{\lambda}^2}2}\paren{\exp\paren{\lambda a^\dagger}a+\sqr{a,\exp\paren{\lambda a^\dagger}}}|0\rangle\\
  =&\exp\paren{-\frac{\abs{\lambda}^2}2}\sqr{a,a^\dagger}\lambda\exp\paren{\lambda a^\dagger}|0\rangle\\
  =&\lambda|\lambda\rangle
}
\subsection{}
$x=z_0\paren{a+a^\dagger}$, $p=\ui\dfrac{\hbar}{2z_0}\paren{a^\dagger-a}$
\eqar{
  \angl{x}=&z_0\angl{a+a^\dagger}\\
  =&z_0\paren{\lambda+\lambda^*}\\
  \angl{x^2}=&z_0^2\angl{\paren{a+a^\dagger}^2}\\
  =&z_0^2\angl{a^2+{a^\dagger}^2+aa^\dagger+a^\dagger a}\\
  =&z_0^2\angl{a^2+{a^\dagger}^2+2a^\dagger a+1}\\
  =&z_0^2\angl{\paren{\lambda+\lambda^*}^2+1}\\
  \angl{\Delta x^2}=&\angl{x^2}-\angl{x}^2\\
  =&z_0^2
  \intertext{}
  \angl{p}=&\ui\dfrac{\hbar}{2z_0}\angl{a^\dagger-a}\\
  =&\ui\dfrac{\hbar}{2z_0}\paren{\lambda^*-\lambda}\\
  \angl{p^2}=&-\dfrac{\hbar^2}{4z_0^2}\angl{\paren{a^\dagger-a}^2}\\
  =&-\dfrac{\hbar^2}{4z_0^2}\angl{a^2+{a^\dagger}^2-aa^\dagger-a^\dagger a}\\
  =&-\dfrac{\hbar^2}{4z_0^2}\angl{a^2+{a^\dagger}^2-2a^\dagger a-1}\\
  =&-\dfrac{\hbar^2}{4z_0^2}\angl{\paren{\lambda^*-\lambda}^2-1}\\
  \angl{\Delta p^2}=&\angl{p^2}-\angl{p}^2\\
  =&\dfrac{\hbar^2}{4z_0^2}\\
  \angl{\Delta p^2}\angl{\Delta x^2}=&\dfrac{\hbar^2}{4}\\
  =&\frac14\abs{\angl{\sqr{x,p}}}^2
}
\subsection{}
\eqar{
  |\lambda\rangle=&\exp\paren{-\frac{\abs{\lambda}^2}2}\exp\paren{\lambda a^\dagger}|0\rangle\\
  =&\exp\paren{-\frac{\abs{\lambda}^2}2}\sum_{n=0}^\infty\frac{\paren{\lambda a^\dagger}^n}{n!}|0\rangle\\
  =&\exp\paren{-\frac{\abs{\lambda}^2}2}\sum_{n=0}^\infty\frac{\lambda^n}{\sqrt{n!}}|n\rangle
  \intertext{Therefore,}
  P\paren{n}=&\exp^{-\abs{\lambda}^2}\frac{\abs{\lambda}^{2n}}{n!}\\
  \sqr{n}_{av}=&\sum_{n=0}^\infty nP\paren{n}\\
  =&\exp^{-\abs{\lambda}^2}\sum_{n=0}^\infty \frac{\abs{\lambda}^{2n}}{(n-1)!}\\
  =&\abs{\lambda}^2\exp^{-\abs{\lambda}^2}\sum_{n=0}^\infty \frac{\abs{\lambda}^{2n}}{n!}\\
  =&\abs{\lambda}^2\\
  \sqr{E_n}_{av}=&\hbar\omega\paren{\abs{\lambda}^{2}+\frac12}
}
\subsection{}
\eqar{
  \angl{n^2}=&\sum_{n=0}^\infty n^2P\paren{n}\\
  =&\exp^{-\abs{\lambda}^2}\sum_{n=0}^\infty n\frac{\abs{\lambda}^{2n}}{(n-1)!}\\
  =&\exp^{-\abs{\lambda}^2}\paren{\sum_{n=0}^\infty \frac{\abs{\lambda}^{2n}}{(n-2)!}+\sum_{n=0}^\infty \frac{\abs{\lambda}^{2n}}{(n-1)!}}\\
  =&\exp^{-\abs{\lambda}^2}\paren{\abs{\lambda}^4\sum_{n=0}^\infty \frac{\abs{\lambda}^{2n}}{n!}+\abs{\lambda}^2\sum_{n=0}^\infty \frac{\abs{\lambda}^{2n}}{n!}}\\
  =&\abs{\lambda}^4+\abs{\lambda}^2\\
  \Delta n=&\abs{\lambda}\\
  \Delta E=&\hbar\omega\abs{\lambda}\\
  \frac{\Delta E}{\sqr{E_n}_{av}}=&\frac{1}{\abs{\lambda}}
}
so the relative uncertainty goes to $0$ at large $n$ limit.
\section{}
In a homogeneous field $B_0$ the $x$ magnetization is
\eqar{
  M_x=&M_0\cos\omega_0t
}
where $\omega_0=\dfrac{2\mu_eB_0}{\hbar}$ is the Larmor frequency.\\
In a non-homogenous field, assuming the initial local magnetization is position
independent
\eqar{
  M_x=&M_0\int\cos\paren{\dfrac{2\mu_eBt}{\hbar}}p(B)\ud B
}
\subsection{}
$p(B)=\dfrac{1}{2a}$
\eqar{
  M_x=&\frac{M_0}{2a}\int^{B_0+a}_{B_0-a}\cos\paren{\dfrac{2\mu_eBt}{\hbar}}\ud B\\
  =&\dfrac{\hbar M_0}{4\mu_eat}\left.\sin\paren{\dfrac{2\mu_eBt}{\hbar}}\right|^{B_0+a}_{B_0-a}\\
  =&\dfrac{\hbar M_0}{4\mu_eat}\paren{\sin\paren{\dfrac{2\mu_et(B_0+a)}{\hbar}}-\sin\paren{\dfrac{2\mu_et(B_0-a)}{\hbar}}}\\
  =&\dfrac{\hbar M_0}{2\mu_eat}\cos\paren{\dfrac{2\mu_etB_0}{\hbar}}\sin\paren{\dfrac{2\mu_eta}{\hbar}}
}
\subsection{}
$p(B)=\dfrac{1}{\sqrt{\pi}a}\ue^{-\paren{B-B_0}^2/a^2}$
\eqar{
  M_x=&\dfrac{M_0}{\sqrt{\pi}a}\int\cos\paren{\dfrac{2\mu_eBt}{\hbar}}\ue^{-\paren{B-B_0}^2/a^2}\ud B\\
  =&\dfrac{M_0}{\sqrt{\pi}a}\Re\paren{\int\exp\paren{-\paren{B-B_0}^2/a^2+\ui\dfrac{2\mu_eBt}{\hbar}}\ud B}\\
  =&\dfrac{M_0}{\sqrt{\pi}}\Re\paren{\int\exp\paren{-x^2+\ui\dfrac{2\mu_eat}{\hbar}x+\ui\dfrac{2\mu_eB_0t}{\hbar}}\ud x}\\
  =&\dfrac{M_0}{\sqrt{\pi}}\Re\paren{\int\exp\paren{-x^2+\ui\dfrac{2\mu_eat}{\hbar}x+\paren{\dfrac{\mu_eat}{\hbar}}^2-\paren{\dfrac{\mu_eat}{\hbar}}^2+\ui\dfrac{2\mu_eB_0t}{\hbar}}\ud x}\\
  =&M_0\Re\paren{\exp\paren{-\paren{\dfrac{\mu_eat}{\hbar}}^2+\ui\dfrac{2\mu_eB_0t}{\hbar}}}\\
  =&M_0\cos\paren{\dfrac{2\mu_eB_0t}{\hbar}}\exp\paren{-\paren{\dfrac{\mu_eat}{\hbar}}^2}
}
\subsection{}
$p(B)=\dfrac{1}{\pi a}\dfrac{1}{1+\paren{B-B_0}^2/a^2}$
\eqar{
  M_x=&\dfrac{M_0}{\pi a}\Re\paren{\int\exp\paren{\ui\dfrac{2\mu_eBt}{\hbar}}\dfrac{1}{1+\paren{B-B_0}^2/a^2}\ud B}\\
  =&\dfrac{M_0}{\pi a}\Re\paren{
    2\pi\ui\mathbf{Res}_{B=B_0+\ui a}\paren{\exp\paren{\ui\dfrac{2\mu_eBt}{\hbar}}\dfrac{1}{1+\paren{B-B_0}^2/a^2}}
  }\\
  =&M_0\cos\paren{\dfrac{2\mu_eB_0t}{\hbar}}\exp\paren{-\dfrac{2\mu_eat}{\hbar}}
}
\subsection{}
$M_x\paren{t}$ is a damped oscillation with a profile corresponds to the Fourier transformation of the $B$ field distribution.
\section{}
The probability is
\eqar{
  \abs{\langle u|\chi\rangle}^2=&\abs{c_1}^2\\
  \abs{\langle u|\chi\rangle}^2=&\langle u|\chi\rangle\langle \chi|u\rangle\\
  =&\langle u|\rho|u\rangle\\
  =&\Tr\paren{\langle u|\rho|u\rangle}\\
  =&\Tr\paren{\rho|u\rangle\langle u|}
}
\section{}
\subsection{}
$|\psi\rangle=c_a\ue^{-\ui E_at/\hbar}|a\rangle+c_b\ue^{-\ui E_bt/\hbar}|b\rangle$
\eqar{
  \ui\hbar\diff{}{t}|\psi\rangle=&\ui\hbar\diff{c_a}{t}\ue^{-\ui E_at/\hbar}|a\rangle+\ui\hbar\diff{c_b}{t}\ue^{-\ui E_bt/\hbar}|b\rangle+E_ac_a\ue^{-\ui E_at/\hbar}|a\rangle+E_bc_b\ue^{-\ui E_bt/\hbar}|b\rangle\\
  H|\psi\rangle=&c_a\ue^{-\ui E_at/\hbar}E_a|a\rangle+c_b\ue^{-\ui E_bt/\hbar}E_b|b\rangle+c_a\ue^{-\ui E_at/\hbar}H'|a\rangle+c_b\ue^{-\ui E_bt/\hbar}H'|b\rangle\\
  &\ui\hbar\diff{c_a}{t}|a\rangle+\ui\hbar\diff{c_b}{t}\ue^{-\ui\omega_0t}|b\rangle\\
  =&c_aH'|a\rangle+c_b\ue^{-\ui\omega_0t}H'|b\rangle
  \intertext{When $H'_{aa}=H'_{bb}=0$ we can remove all operators from the equation}
  &\ui\hbar\diff{c_a}{t}|a\rangle+\ui\hbar\diff{c_b}{t}\ue^{-\ui\omega_0t}|b\rangle\\
  =&c_aH'_{ba}|b\rangle+c_b\ue^{-\ui\omega_0t}H'_{ab}|a\rangle
  \intertext{Left multiply by $\rangle a|$ or $\rangle b|$,}
  \diff{c_a}{t}=&-\frac{\ui}{\hbar}c_b\ue^{-\ui\omega_0t}H'_{ab}\\
  \diff{c_b}{t}=&-\frac{\ui}{\hbar}c_a\ue^{\ui\omega_0t}H'_{ba}
}
\subsection{}
Zeroth order
\eqar{
  c_a^{(0)}=&1\\
  c_b^{(0)}=&0
  \intertext{First order}
  \diff{c_a^{(1)}}{t}=&-\frac{\ui}{\hbar}c_b^{(0)}\ue^{-\ui\omega_0t}H'_{ab}\\
  =&0\\
  \diff{c_b^{(1)}}{t}=&-\frac{\ui}{\hbar}c_a^{(0)}\ue^{\ui\omega_0t}H'_{ba}\\
  =&-\frac{\ui}{\hbar}\ue^{\ui\omega_0t}H'_{ba}\\
  c_a^{(1)}=&0\\
  c_b^{(1)}=&\int_0^t-\frac{\ui V_0^*}{2\hbar}\paren{\ue^{\ui\paren{\omega+\omega_0}t'}+\ue^{\ui\paren{\omega_0-\omega}t'}}\ud t'\\
  =&-\frac{V_0^*}{2\hbar}\paren{\frac{\ue^{\ui\paren{\omega+\omega_0}t}-1}{\omega+\omega_0}+\frac{\ue^{\ui\paren{\omega_0-\omega}t}-1}{\omega_0-\omega}}
  \intertext{When $\abs{\omega_0-\omega}\ll\omega_0+\omega$}
  c_b^{(1)}\approx&-\frac{V_0^*}{2\hbar}\frac{\ue^{\ui\paren{\omega_0-\omega}t}-1}{\omega_0-\omega}\\
  \abs{c_b^{(1)}}^2\approx&\frac{\abs{V_0}^2}{4\hbar^2}\frac{\abs{\ue^{\ui\paren{\omega_0-\omega}t/2}-\ue^{-\ui\paren{\omega_0-\omega}t/2}}^2}{\paren{\omega_0-\omega}^2}\\
  =&\frac{\abs{V_0}^2}{\hbar^2}\frac{\sin^2\paren{\paren{\omega_0-\omega}t/2}}{\paren{\omega_0-\omega}^2}
}
\subsection{}
Right. I guess there isn't anything to solve for this one?
\subsection{}
\eqar{
  \diff{c_a}{t}=&-\frac{\ui V_0}{2\hbar}c_b\ue^{-\ui\paren{\omega_0-\omega}t}\\
  =&-\frac{\ui\Omega}{2}c_b\ue^{-\ui\paren{\omega_0-\omega}t}\\
  \diff{c_b}{t}=&-\frac{\ui V_0^*}{2\hbar}c_a\ue^{\ui\paren{\omega_0-\omega}t}\\
  =&-\frac{\ui\Omega^*}{2}c_a\ue^{\ui\paren{\omega_0-\omega}t}
  \intertext{where $\Omega=\dfrac{V_0}{\hbar}$}
  \diff[2]{c_b}{t}=&-\frac{\ui\Omega^*}{2}\diff{c_a}{t}\ue^{\ui\paren{\omega_0-\omega}t}+\frac{\Omega^*\paren{\omega_0-\omega}}{2}c_a\ue^{\ui\paren{\omega_0-\omega}t}\\
  =&-\frac{\abs{\Omega}^2}{4}c_b+\ui\paren{\omega_0-\omega}\diff{c_b}{t}\\
  0=&\diff[2]{c_b}{t}-\ui\paren{\omega_0-\omega}\diff{c_b}{t}+\frac{\abs{\Omega}^2}{4}c_b
  \intertext{For $c_b\propto\ue^{\ui\omega't}$}
  \omega'=&-\frac\delta2\pm\omega_R
  \intertext{where $\delta=\omega-\omega_0$. Since $c_b(0)=0$}
  c_b=&c_{b0}\ue^{-\ui\delta t/2}\sin\omega_Rt\\
  c_a=&\frac{2\ui}{\Omega^*}\ue^{\ui\delta t}\diff{c_b}{t}\\
  =&\frac{2\ui c_{b0}}{\Omega^*}\ue^{\ui\delta t/2}\paren{-\ui\frac{\delta}{2}\sin\omega_Rt+\omega_R\cos\omega_Rt}
  \intertext{Since $c_a(0)=1$}
  c_b=&-\ui\frac{\Omega^*}{2\omega_R}\ue^{-\ui\delta t/2}\sin\omega_Rt\\
  c_a=&\ue^{\ui\delta t/2}\paren{\cos\omega_Rt-\ui\frac{\delta}{2\omega_R}\sin\omega_Rt}
}
\subsection{}
Transtion probability
\eqar{
  P_{a\rightarrow b}=&\abs{c_b}^2\\
  =&\frac{\abs{\Omega}^2}{4\omega_R^2}\sin^2\omega_Rt
}
Since $\abs\Omega\leqslant2\omega_R$ and $\sin^2\omega_Rt\leqslant1$, $P_{a\rightarrow b}\leqslant1$.
\eqar{
  \abs{c_a}^2+\abs{c_b}^2=&\frac{\abs{\Omega}^2}{4\omega_R^2}\sin^2\omega_Rt+\cos^2\omega_Rt+\frac{\delta^2}{4\omega_R^2}\sin^2\omega_Rt\\
  =&\sin^2\omega_Rt+\cos^2\omega_Rt\\
  =&1
}
\subsection{}
For $\omega_Rt\ll1$ (meaning of small)
\eqar{
  P_{a\rightarrow b}\approx&\frac{\abs{\Omega}^2}{4\omega_R^2}\omega_R^2t^2\\
  =&\frac{\abs{\Omega}^2t^2}{4}
}
\subsection{}
The system comes back to the original state after $\frac{2\pi}{\omega_R}$.
\section{}
\subsection{}
Initial state,
\eqar{
  |i\rangle=&|b;0\rangle
  \intertext{Final state,}
  |f\rangle=&|a;\vec k,s\rangle
}
where $\vec k$ and $s$ are the wave vector and polarization of the photon.
Matrix element
\eqar{
  \langle f|H_{E2}|i\rangle=&-q\langle f|\paren{\vec r\cdot\vec E_0}\paren{\vec k\cdot\vec r}|i\rangle\\
  =&\ui \sqrt{\frac{\hbar\omega_k}{2\varepsilon_0V}}q\langle a;\vec k,s|\paren{\vec r\cdot\vec n}a^\dagger_{k,s}\paren{\vec k\cdot\vec r}|b;0\rangle\\
  =&\ui q\sqrt{\frac{\hbar\omega_k}{2\varepsilon_0V}}\langle a|\paren{\vec r\cdot\vec n}\paren{\vec k\cdot\vec r}|b\rangle\\
  R_{b\rightarrow a}=&\frac{\hbar\omega_kq^2}{2\varepsilon_0V}\abs{\langle a|\paren{\vec r\cdot\vec n}\paren{\vec k\cdot\vec r}|b\rangle}^2
}
Note that the coefficient still depend on the normalization volumn $V$ which should disappear after taking the sum over a finite range of $\vec k$.
\subsection{}
The matrix element is symmetric for rotation around $\vec k$ therefore it is the
same after rotating around $\vec k$ by $\pi$. Since $\vec k\cdot\vec n=0$
\eqar{
  \langle 1s|\paren{\vec r\cdot\vec n}\paren{\vec k\cdot\vec r}|2s\rangle=&\langle 1s|\paren{-\vec r\cdot\vec n}\paren{\vec k\cdot\vec r}|2s\rangle\\
  \langle 1s|\paren{\vec r\cdot\vec n}\paren{\vec k\cdot\vec r}|2s\rangle=&0
}
\end{document}
