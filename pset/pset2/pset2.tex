\documentclass[10pt,fleqn]{article}
\newcommand{\name}[1]{\def\psettitlename{#1}}
\newcommand{\course}[1]{\def\psettitlecourse{#1}}
\newcommand{\rsection}[1]{\def\psettitlersection{#1}}
\newcommand{\psetnum}[1]{\def\psettitlepsetnum{#1}}
% \usepackage[journal=rsc]{chemstyle}
% \usepackage{mhchem}
\usepackage{amsmath}
\usepackage{amssymb}
\usepackage{amsfonts}
\usepackage{esint}
\usepackage{bbm}
\usepackage{amscd}
\usepackage{picinpar}
\usepackage[pdftex]{graphicx}
\usepackage{indentfirst}
\usepackage{wrapfig}
\usepackage{units}
\usepackage{textcomp}
\usepackage[utf8x]{inputenc}
\usepackage{feyn}
\usepackage{feynmp}
\DeclareGraphicsRule{*}{mps}{*}{}
\newcommand{\ud}{\mathrm{d}}
\newcommand{\ue}{\mathrm{e}}
\newcommand{\ui}{\mathrm{i}}
\newcommand{\res}{\mathrm{Res}}
\newcommand{\Tr}{\mathrm{Tr}}
\newcommand{\dsum}{\displaystyle\sum}
\newcommand{\dprod}{\displaystyle\prod}
\newcommand{\dlim}{\displaystyle\lim}
\newcommand{\dint}{\displaystyle\int}
\newcommand{\fsno}[1]{{\!\not\!{#1}}}
\newcommand{\eqar}[1]
{
  \begin{align*}
    #1
  \end{align*}
}
\newcommand{\texp}[2]{\ensuremath{{#1}\times10^{#2}}}
\newcommand{\dexp}[2]{\ensuremath{{#1}\cdot10^{#2}}}
\newcommand{\eval}[2]{{\left.{#1}\right|_{#2}}}
\newcommand{\paren}[1]{{\left({#1}\right)}}
\newcommand{\lparen}[1]{{\left({#1}\right.}}
\newcommand{\rparen}[1]{{\left.{#1}\right)}}
\newcommand{\abs}[1]{{\left|{#1}\right|}}
\newcommand{\sqr}[1]{{\left[{#1}\right]}}
\newcommand{\crly}[1]{{\left\{{#1}\right\}}}
\newcommand{\angl}[1]{{\left\langle{#1}\right\rangle}}
\newcommand{\tpdiff}[4][{}]{{\paren{\frac{\partial^{#1} {#2}}{\partial {#3}{}^{#1}}}_{#4}}}
\newcommand{\tpsdiff}[4][{}]{{\paren{\frac{\partial^{#1}}{\partial {#3}{}^{#1}}{#2}}_{#4}}}
\newcommand{\pdiff}[3][{}]{{\frac{\partial^{#1} {#2}}{\partial {#3}{}^{#1}}}}
\newcommand{\diff}[3][{}]{{\frac{\ud^{#1} {#2}}{\ud {#3}{}^{#1}}}}
\newcommand{\psdiff}[3][{}]{{\frac{\partial^{#1}}{\partial {#3}{}^{#1}} {#2}}}
\newcommand{\sdiff}[3][{}]{{\frac{\ud^{#1}}{\ud {#3}{}^{#1}} {#2}}}
\newcommand{\tpddiff}[4][{}]{{\left(\dfrac{\partial^{#1} {#2}}{\partial {#3}{}^{#1}}\right)_{#4}}}
\newcommand{\tpsddiff}[4][{}]{{\paren{\dfrac{\partial^{#1}}{\partial {#3}{}^{#1}}{#2}}_{#4}}}
\newcommand{\pddiff}[3][{}]{{\dfrac{\partial^{#1} {#2}}{\partial {#3}{}^{#1}}}}
\newcommand{\ddiff}[3][{}]{{\dfrac{\ud^{#1} {#2}}{\ud {#3}{}^{#1}}}}
\newcommand{\psddiff}[3][{}]{{\frac{\partial^{#1}}{\partial{}^{#1} {#3}} {#2}}}
\newcommand{\sddiff}[3][{}]{{\frac{\ud^{#1}}{\ud {#3}{}^{#1}} {#2}}}
\usepackage{fancyhdr}
\usepackage{multirow}
\usepackage{fontenc}
% \usepackage{tipa}
\usepackage{ulem}
\usepackage{color}
\usepackage{cancel}
\newcommand{\hcancel}[2][black]{\setbox0=\hbox{#2}%
  \rlap{\raisebox{.45\ht0}{\textcolor{#1}{\rule{\wd0}{1pt}}}}#2}
\pagestyle{fancy}
\setlength{\headheight}{67pt}
\fancyhead{}
\fancyfoot{}
\fancyfoot[C]{\thepage}
\fancyhead[R]
{
  \psettitlename \\
  \psettitlecourse{} Problem Set \psettitlepsetnum \\
  \ifx\psettitlersection\empty
  \else
  Recitation Section \psettitlersection
  \fi
}
\renewcommand{\footruleskip}{0pt}
\renewcommand{\headrulewidth}{0.4pt}
\renewcommand{\footrulewidth}{0pt}
\addtolength{\hoffset}{-1.3cm}
\addtolength{\voffset}{-2cm}
\addtolength{\textwidth}{3cm}
\addtolength{\textheight}{2.5cm}
\renewcommand{\footskip}{10pt}
\setlength{\headwidth}{\textwidth}
\setlength{\headsep}{20pt}
\setlength{\marginparwidth}{0pt}
\parindent=0pt
\psetnum{2}
\course{Physics 251b}
\rsection{1}
\name{Yichao Yu}
\renewcommand{\thesection}{\arabic{section}.}
\renewcommand{\thesubsection}{(\alph{subsection})}
\renewcommand{\thesubsubsection}{\roman{subsubsection}.}

\begin{document}
\section{}
\eqar{
  |1,1;2,2\rangle=&|1,1\rangle_1|1,1\rangle_2\\
  |1,1;2,-2\rangle=&|1,-1\rangle_1|1,-1\rangle_2\\
  |1,1;2,1\rangle=&\frac{J_-}{2\hbar}|1,1;2,2\rangle\\
  =&\frac{L_{1-}+L_{2-}}{2\hbar}|1,1\rangle_1|1,1\rangle_2\\
  =&\frac{|1,0\rangle_1|1,1\rangle_2+|1,1\rangle_1|1,0\rangle_2}{\sqrt{2}}\\
  |1,1;2,-1\rangle=&\frac{J_+}{2\hbar}|1,1;2,2\rangle\\
  =&\frac{L_{1+}+L_{2+}}{2\hbar}|1,-1\rangle_1|1,-1\rangle_2\\
  =&\frac{|1,0\rangle_1|1,-1\rangle_2+|1,-1\rangle_1|1,0\rangle_2}{\sqrt{2}}\\
  |1,1;2,0\rangle=&\frac{J_-}{\sqrt{6}\hbar}|1,1;2,1\rangle\\
  =&\frac{L_{1-}+L_{2-}}{\sqrt{6}\hbar}\frac{|1,0\rangle_1|1,1\rangle_2+|1,1\rangle_1|1,0\rangle_2}{\sqrt{2}}\\
  =&\frac{|1,-1\rangle_1|1,1\rangle_2+2|1,0\rangle_1|1,0\rangle_2+|1,1\rangle_1|1,-1\rangle_2}{\sqrt{6}}
  \intertext{From orthogonality}
  |1,1;1,1\rangle=&\frac{|1,0\rangle_1|1,1\rangle_2-|1,1\rangle_1|1,0\rangle_2}{\sqrt{2}}\\
  |1,1;1,-1\rangle=&\frac{|1,0\rangle_1|1,-1\rangle_2-|1,-1\rangle_1|1,0\rangle_2}{\sqrt{2}}\\
  |1,1;1,0\rangle=&\frac{J_-}{\sqrt{2}\hbar}|1,1;1,1\rangle\\
  =&\frac{L_{1-}+L_{2-}}{\sqrt{2}\hbar}\frac{|1,0\rangle_1|1,1\rangle_2-|1,1\rangle_1|1,0\rangle_2}{\sqrt{2}}\\
  =&\frac{|1,-1\rangle_1|1,1\rangle_2-|1,1\rangle_1|1,-1\rangle_2}{\sqrt{2}}
  \intertext{From orthogonality}
  |1,1;0,0\rangle=&\frac{|1,-1\rangle_1|1,1\rangle_2-|1,0\rangle_1|1,0\rangle_2+|1,1\rangle_1|1,-1\rangle_2}{\sqrt{3}}
}
\section{}
\subsection{}
Since $J_z=L_z+S_z$ and $\sqr{L_z, S_z}=0$, the $J_z$ eigenstates are sum of
$L_z$ and $S_z$ eigenstates that has the same sum of $m_l$ and $m_s$.
(The only variable on the RHS are $m_l$ and $m_s$ among which $m_s$ can only be
$\pm\dfrac12$ so the constraint above limit the decomposition to the given form.)
\subsection{}
Hamiltonian
\eqar{
  H=&\frac{L^2}{2ma^2}+V_0+\frac{e^2\vec L\cdot\vec S}{2mc^2a^3}\\
  =&\frac{L^2}{2ma^2}+V_0+\frac{e^2}{4mc^2a^3}\paren{J^2-L^2-S^2}\\
  =&\frac{l\paren{l+1}}{2ma^2}+V_0+\frac{e^2}{4mc^2a^3}\paren{j\paren{j+1}-l\paren{l+1}-\frac34}
}
When $l=0$ there's only one manifold instead of two.
\section{}
\eqar{
  \ue^{\lambda A}B\ue^{-\lambda A}=&\paren{\sum_{m=0}^\infty\frac{\lambda^mA^m}{m!}}B\paren{\sum_{n=0}^\infty\frac{\lambda^n\paren{-A}^n}{n!}}\\
  =&\sum_{m,n=0}^\infty\frac{\lambda^mA^m}{m!}B\frac{\lambda^n\paren{-A}^n}{n!}\\
  =&\sum_{m=0}^\infty\sum_{n=0}^m\frac{\lambda^mA^{m-n}B\paren{-A}^n}{\paren{m-n}!n!}\\
  =&\sum_{m=0}^\infty\frac{\lambda^m}{m!}\sum_{n=0}^m\frac{m!A^{m-n}B\paren{-A}^n}{\paren{m-n}!n!}
}
So now we just need to show that
\eqar{
  \sqr{A^{(m)}, B}=&\sum_{n=0}^m\frac{m!A^{m-n}B\paren{-A}^n}{\paren{m-n}!n!}
}
where $\sqr{A^{(0)}, B}\equiv B$,
$\sqr{A^{(n)}, B}\equiv\sqr{A, \sqr{A^{(n-1)}, B}}$.
For $m=0$, this is trivially true. If it is true for $m-1$,
\eqar{
  \sqr{A^{(m)}, B}=&\sqr{A, \sum_{n=0}^{m-1}\frac{\paren{m-1}!A^{m-n-1}B\paren{-A}^n}{\paren{m-n-1}!n!}}\\
  =&\sum_{n=0}^{m-1}\frac{\paren{m-1}!A^{m-n}B\paren{-A}^n}{\paren{m-n-1}!n!}+\sum_{n=1}^{m}\frac{\paren{m-1}!A^{m-n}B\paren{-A}^{n}}{\paren{m-n}!\paren{n-1}!}
  \intertext{Extend the summation taking advantage of $(-1)!=\infty$}
  \sqr{A^{(m)}, B}=&\sum_{n=0}^{m}\frac{\paren{m-1}!A^{m-n}B\paren{-A}^n}{\paren{m-n-1}!n!}+\frac{\paren{m-1}!A^{m-n}B\paren{-A}^{n}}{\paren{m-n}!\paren{n-1}!}\\
  =&\sum_{n=0}^{m}\frac{\paren{m}!A^{m-n}B\paren{-A}^n}{\paren{m-n}!n!}
  \paren{\frac{m-n}{m}+\frac{n}{m}}\\
  =&\sum_{n=0}^{m}\frac{\paren{m}!A^{m-n}B\paren{-A}^n}{\paren{m-n}!n!}
}
Therefore, the equation is true for all $m\geqslant0$.\\
If $\sqr{A,B}=\gamma B$, $\sqr{A^{(n)},B}=\gamma^nB$
\eqar{
  \ue^{\lambda A}B\ue^{-\lambda A}=&\sum_{m=0}^\infty\frac{\lambda^m}{m!}\sqr{A^{(m)}, B}\\
  =&\sum_{m=0}^\infty\frac{\lambda^m\gamma^m}{m!}B\\
  =&\ue^{\lambda\gamma}B
}
\section{}
Since $\sqr{\lambda, A}=0$ and $\sqr{A, A}=0$ (i.e. everything commutes),
\eqar{
  \diff{\ue^{\lambda A}}{\lambda}=&A\ue^{\lambda A}
}
We have
\eqar{
  \diff{G}{\lambda}=&\diff{\ue^{\lambda A}\ue^{\lambda B}}{\lambda}\\
  =&\ue^{\lambda A}\diff{\ue^{\lambda B}}{\lambda}+\diff{\ue^{\lambda A}}{\lambda}\ue^{\lambda B}\\
  =&\ue^{\lambda A}B\ue^{\lambda B}+A\ue^{\lambda A}\ue^{\lambda B}\\
  =&\ue^{\lambda A}B\ue^{-\lambda A}\ue^{\lambda A}\ue^{\lambda B}+A\ue^{\lambda A}\ue^{\lambda B}\\
  =&\paren{A+\ue^{\lambda A}B\ue^{-\lambda A}}G\\
  =&\paren{A+\sum_{m=0}^\infty\frac{\lambda^m}{m!}\sqr{A^{(m)}, B}}G
}
Using this,
\eqar{
  \diff{G}{\lambda}=&\paren{\diff{\ue^{\lambda B^\dagger}\ue^{\lambda A^\dagger}}{\lambda}}^\dagger\\
  =&\paren{\paren{B^\dagger+\sum_{m=0}^\infty\frac{\lambda^m}{m!}\sqr{{B^\dagger}^{(m)}, A^\dagger}}G^\dagger}^\dagger\\
  =&G\paren{B+\sum_{m=0}^\infty\frac{\lambda^m}{m!}\sqr{A, B^{(m)}}}
}
If $C\equiv\sqr{A,B}$ commutes with both $A$ and $B$ (and therefore $A+B$),
\eqar{
  \diff{G}{\lambda}=&\paren{A+B+\lambda C}G\\
  G=&\ue^{A+B+\lambda^2C/2}\\
  \ue^{A}\ue^{B}=&\ue^{A+B+C/2}
}
\section{}
\subsection{}
Eigenvalue $\lambda$
\eqar{
  0=&(h-\lambda)^2 - \abs{g}^2\\
  h-\lambda=&\pm\abs{g}\\
  \lambda=&h\pm\abs{g}
}
Corresponding eigen vectors are $\dfrac1{\sqrt{2}}\paren{1, \pm\dfrac{g}{\abs{g}}}$

\subsection{}
Initial state
\eqar{
  |\psi_0\rangle=&\dfrac1{\sqrt{2}}\paren{|+\rangle+|-\rangle}
  \intertext{At time $t$,}
  |\psi_t\rangle=&\dfrac1{\sqrt{2}}\paren{\exp\paren{-\ui\frac{h+\abs{g}}{\hbar}t}|+\rangle+\exp\paren{-\ui\frac{h-\abs{g}}{\hbar}t}|-\rangle}\\
  =&\dfrac{\ue^{-\ui ht/\hbar}}{\sqrt{2}}\paren{\ue^{-\ui\abs{g}t/\hbar}|+\rangle+\ue^{\ui\abs{g}t/\hbar}|-\rangle}\\
  =&\dfrac{\ue^{-\ui ht/\hbar}}{2}\paren{\ue^{-\ui\abs{g}t/\hbar}\paren{|1\rangle+\dfrac{g}{\abs{g}}|2\rangle}+\ue^{\ui\abs{g}t/\hbar}\paren{|1\rangle-\dfrac{g}{\abs{g}}|2\rangle}}\\
  =&\ue^{-\ui ht/\hbar}\paren{\cos\paren{\frac{\abs{g}t}{\hbar}}|1\rangle-\ui\dfrac{g}{\abs{g}}\sin\paren{\frac{\abs{g}t}{\hbar}}|2\rangle}
}
\section{}
\subsection{}
\subsection{}
\end{document}
