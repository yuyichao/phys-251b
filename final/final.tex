\documentclass[10pt,fleqn]{article}
% \usepackage[journal=rsc]{chemstyle}
% \usepackage{mhchem}
\usepackage{amsmath}
\usepackage{amssymb}
\usepackage{amsfonts}
\usepackage{esint}
\usepackage{bbm}
\usepackage{amscd}
\usepackage{picinpar}
\usepackage[pdftex]{graphicx}
\usepackage{indentfirst}
\usepackage{wrapfig}
\usepackage{units}
\usepackage{textcomp}
\usepackage[utf8x]{inputenc}
\usepackage{feyn}
\usepackage{feynmp}
\DeclareGraphicsRule{*}{mps}{*}{}
\newcommand{\ud}{\mathrm{d}}
\newcommand{\ue}{\mathrm{e}}
\newcommand{\ui}{\mathrm{i}}
\newcommand{\res}{\mathrm{Res}}
\newcommand{\Tr}{\mathrm{Tr}}
\newcommand{\dsum}{\displaystyle\sum}
\newcommand{\dprod}{\displaystyle\prod}
\newcommand{\dlim}{\displaystyle\lim}
\newcommand{\dint}{\displaystyle\int}
\newcommand{\fsno}[1]{{\!\not\!{#1}}}
\newcommand{\eqar}[1]
{
  \begin{align*}
    #1
  \end{align*}
}
\newcommand{\texp}[2]{\ensuremath{{#1}\times10^{#2}}}
\newcommand{\dexp}[2]{\ensuremath{{#1}\cdot10^{#2}}}
\newcommand{\eval}[2]{{\left.{#1}\right|_{#2}}}
\newcommand{\paren}[1]{{\left({#1}\right)}}
\newcommand{\lparen}[1]{{\left({#1}\right.}}
\newcommand{\rparen}[1]{{\left.{#1}\right)}}
\newcommand{\abs}[1]{{\left|{#1}\right|}}
\newcommand{\sqr}[1]{{\left[{#1}\right]}}
\newcommand{\crly}[1]{{\left\{{#1}\right\}}}
\newcommand{\angl}[1]{{\left\langle{#1}\right\rangle}}
\newcommand{\tpdiff}[4][{}]{{\paren{\frac{\partial^{#1} {#2}}{\partial {#3}{}^{#1}}}_{#4}}}
\newcommand{\tpsdiff}[4][{}]{{\paren{\frac{\partial^{#1}}{\partial {#3}{}^{#1}}{#2}}_{#4}}}
\newcommand{\pdiff}[3][{}]{{\frac{\partial^{#1} {#2}}{\partial {#3}{}^{#1}}}}
\newcommand{\diff}[3][{}]{{\frac{\ud^{#1} {#2}}{\ud {#3}{}^{#1}}}}
\newcommand{\psdiff}[3][{}]{{\frac{\partial^{#1}}{\partial {#3}{}^{#1}} {#2}}}
\newcommand{\sdiff}[3][{}]{{\frac{\ud^{#1}}{\ud {#3}{}^{#1}} {#2}}}
\newcommand{\tpddiff}[4][{}]{{\left(\dfrac{\partial^{#1} {#2}}{\partial {#3}{}^{#1}}\right)_{#4}}}
\newcommand{\tpsddiff}[4][{}]{{\paren{\dfrac{\partial^{#1}}{\partial {#3}{}^{#1}}{#2}}_{#4}}}
\newcommand{\pddiff}[3][{}]{{\dfrac{\partial^{#1} {#2}}{\partial {#3}{}^{#1}}}}
\newcommand{\ddiff}[3][{}]{{\dfrac{\ud^{#1} {#2}}{\ud {#3}{}^{#1}}}}
\newcommand{\psddiff}[3][{}]{{\frac{\partial^{#1}}{\partial{}^{#1} {#3}} {#2}}}
\newcommand{\sddiff}[3][{}]{{\frac{\ud^{#1}}{\ud {#3}{}^{#1}} {#2}}}
\usepackage{fancyhdr}
\usepackage{multirow}
\usepackage{fontenc}
% \usepackage{tipa}
\usepackage{ulem}
\usepackage{color}
\usepackage{cancel}
\newcommand{\hcancel}[2][black]{\setbox0=\hbox{#2}%
  \rlap{\raisebox{.45\ht0}{\textcolor{#1}{\rule{\wd0}{1pt}}}}#2}
\pagestyle{fancy}
\setlength{\headheight}{67pt}
\fancyhead{}
\fancyfoot{}
\fancyfoot[C]{\thepage}
\fancyhead[R]
{
  Yichao Yu\\
  Physics 251b Final\\
}
\renewcommand{\footruleskip}{0pt}
\renewcommand{\headrulewidth}{0.4pt}
\renewcommand{\footrulewidth}{0pt}
\addtolength{\hoffset}{-1.3cm}
\addtolength{\voffset}{-2cm}
\addtolength{\textwidth}{3cm}
\addtolength{\textheight}{2.5cm}
\renewcommand{\footskip}{10pt}
\setlength{\headwidth}{\textwidth}
\setlength{\headsep}{20pt}
\setlength{\marginparwidth}{0pt}
\parindent=0pt
\renewcommand{\thesection}{\arabic{section}.}
\renewcommand{\thesubsection}{(\alph{subsection})}
\renewcommand{\thesubsubsection}{\roman{subsubsection}.}
\begin{document}
\section{}
\subsection{}
Communator of each component
\eqar{
  \sqr{L_i+g_0S_i,J_j}=&\sqr{L_i+g_0S_i,L_j+S_j}\\
  =&\sqr{L_i,L_j}+g_0\sqr{S_i,S_j}\\
  =&\ui\hbar\varepsilon_{ijk}\paren{L_k+g_0S_k}\\
  \sqr{L_i+g_0S_i,\hat n\cdot\vec J}=&\ui\hbar\varepsilon_{ijk}n_j\paren{L_k+g_0S_k}\\
  =&\ui\hbar \paren{\hat n\times\paren{\vec L+g_0\vec S}}_i\\
  \sqr{\vec L+g_0\vec S,\hat n\cdot\vec J}=&\ui\hbar\varepsilon_{ijk}n_j\paren{L_k+g_0S_k}\\
  =&\ui\hbar\hat n\times\paren{\vec L+g_0\vec S}
}
Therefore for any $\vec n$
\eqar{
  &\ui\hbar\hat n\times\langle 0|\vec L+g_0\vec S|0\rangle\\
  =&\langle 0|\sqr{L_i+g_0S_i,\hat n\cdot\vec J}|0\rangle\\
  =&\langle 0|\sqr{L_i+g_0S_i,0}|0\rangle\\
  =&0\\
  &\langle 0|\vec L+g_0\vec S|0\rangle\\
  =&0
}
This is a special case of the Wigner-Eckart Theorem because the $|0\rangle$ state
is spherical symmetric. The physical origin of the factor $g_0$ is the low energy
limit of the Dirac equation of electron (and QED corrections on top of it).
\subsection{}
\section{}
\section{}
\subsection{}
Radial component of $\vec j$
\eqar{
  j_r=&\frac{\hbar}{2m\ui}\paren{\psi^*\pdiff{}{r}\psi-\psi\pdiff{}{r}\psi^*}\\
  =&\frac{\hbar}{m}\Im\paren{\psi^*\pdiff{}{r}\psi}
}
The part terms that is due to interference (for $\psi=\psi_1+\psi_2$)
\eqar{
  j_r'=&j_r-j_{r1}-j_{r2}\\
  =&\frac{\hbar}{m}\Im\paren{\psi^*\pdiff{}{r}\psi}-\frac{\hbar}{m}\Im\paren{\psi_1^*\pdiff{}{r}\psi_1}-\frac{\hbar}{m}\Im\paren{\psi_2^*\pdiff{}{r}\psi_2}\\
  =&\frac{\hbar}{m}\Im\paren{\psi_1^*\pdiff{}{r}\psi_2}+\frac{\hbar}{m}\Im\paren{\psi_2^*\pdiff{}{r}\psi_1}
}
Scattering wave function
\eqar{
  \psi=&\ue^{\ui kr\cos\theta}+f\frac{\ue^{\ui kr}}{r}
  \intertext{current density}
  j_r'=&\frac{\hbar}{m}\Im\paren{f\ue^{-\ui kr\cos\theta}\pdiff{}{r}\frac{\ue^{\ui kr}}{r}+f^*\frac{\ue^{-\ui kr}}{r}\pdiff{}{r}\ue^{\ui kr\cos\theta}}\\
  =&\frac{\hbar}{m}\Im\paren{f\ue^{-\ui kr\cos\theta}\frac{r\ui k-1}{r^2}\ue^{\ui kr}+f^*\ui k\cos\theta\frac{\ue^{-\ui kr}}{r}\ue^{\ui kr\cos\theta}}
  \intertext{Ignoring $r^{-2}$ term for large $r$}
  j_r'\approx&\frac{\hbar k}{m}\frac{1}{r}\Im\paren{\ui f\ue^{-\ui kr\cos\theta}\ue^{\ui kr}+\ui f^*\cos\theta\ue^{-\ui kr}\ue^{\ui kr\cos\theta}}\\
  =&\frac{\hbar k}{m}\frac{1}{r}\Im\paren{\ui \ue^{\ui kr\paren{\cos\theta-1}}f^*\cos\theta+\ui \ue^{\ui kr\paren{1-\cos\theta}}f}
}
\subsection{}
\eqar{
  \int_a^b\ud x\ue^{\ui\lambda x}f=&\int_a^bf\ud\frac{\ue^{\ui\lambda x}}{\ui\lambda}\\
  =&\left.\frac{\ue^{\ui\lambda x}f}{\ui\lambda}\right|_a^b-\int_a^b\ud x f'\frac{\ue^{\ui\lambda x}}{\ui\lambda}
  \intertext{Using the same integral by part, we can show that the second term is $O\paren{\dfrac{1}{\lambda^2}}$. Therefore,}
  \int_a^b\ud x\ue^{\ui\lambda x}f=&\frac{\ue^{\ui\lambda b}f(b)-\ue^{\ui\lambda a}f(a)}{\ui\lambda}+O\paren{\dfrac{1}{\lambda^2}}
}
\subsection{}
Total interference current
\eqar{
  J=&r^2\int\ud\Omega\frac{\hbar k}{m}\frac{1}{r}\Im\paren{\ui \ue^{\ui kr\paren{\cos\theta-1}}f^*\cos\theta+\ui \ue^{\ui kr\paren{1-\cos\theta}}f}\\
  =&\frac{\hbar kr}{m}\Re\paren{\int\ud\theta\int\ud\phi\sin\theta\paren{\ue^{\ui kr\paren{\cos\theta-1}}f^*\cos\theta+ \ue^{\ui kr\paren{1-\cos\theta}}f}}\\
  =&\frac{2\pi\hbar kr}{m}\Re\paren{\int_{-1}^1\ud\cos\theta\paren{\ue^{\ui kr\paren{\cos\theta-1}}f^*\cos\theta+ \ue^{\ui kr\paren{1-\cos\theta}}f}}\\
  \approx&\frac{2\pi\hbar kr}{m}\Re\paren{
    \frac{f^*\paren{0}+\ue^{-2\ui kr}f^*\paren{\pi}}{\ui kr}
    -\frac{f\paren{0}-\ue^{2\ui kr}f\paren{\pi}}{\ui kr}
  }\\
  =&\frac{2\pi\hbar}{m}\Im\paren{
    f^*\paren{0}+\ue^{-2\ui kr}f^*\paren{\pi}
    -f\paren{0}+\ue^{2\ui kr}f\paren{\pi}
  }\\
  =&-\frac{4\pi\hbar}{m}\Im\paren{f\paren{0}}
}
\subsection{}
The total current of the incident wave is zero (since it's a plain wave).
The total current of the scatterend wave.
\eqar{
  J_2=&r^2\int\ud\Omega\frac{\hbar}{m}\Im\paren{\psi_r^*\pdiff{}{r}\psi_2}\\
  =&r^2\int\ud\Omega\frac{\hbar}{m}\Im\paren{f^*f\frac{\ue^{-\ui kr}}{r}\pdiff{}{r}\frac{\ue^{\ui kr}}{r}}\\
  =&\int\ud\Omega\frac{\hbar}{m}\Im\paren{f^*f\frac{r\ui k-1}{r}}
  \intertext{Ignore real term in the integral}
  =&\frac{\hbar k}{m}\int\ud\Omega f^*f\\
  =&\frac{\hbar k}{m}\sigma_{tot}
}
Therefore
\eqar{
  0=&\frac{\hbar k}{m}\sigma_{tot}-\frac{4\pi\hbar}{m}\Im\paren{f\paren{0}}\\
  \sigma_{tot}=&\frac{4\pi}{k}\Im\paren{f\paren{0}}
}
\section{}
\subsection{}
In first Born approximation (Use $V_r$ to represent $\partial_rV$)
\eqar{
  f_\sigma=&-\frac{1}{4\pi}\frac{2m}{\hbar^2}\int\ud^3r'\ue^{\ui\vec k\cdot\vec r'}V\paren{r',\sigma}\chi_\sigma\ue^{-\ui\vec k'\cdot\vec r'}\\
  =&-\frac{1}{4\pi}\frac{2m}{\hbar^2}\int\ud^3r'\ue^{\ui\vec k\cdot\vec r'}\paren{-\paren{1+\ui\xi}V(r')+\frac{c}{r'}V_r(r')\vec\sigma\cdot\frac{\vec r'\times\vec p}{\hbar}}\ue^{-\ui\vec k'\cdot\vec r'}\chi_\sigma\\
  =&-\frac{1}{4\pi}\frac{2m}{\hbar^2}\int\ud^3r'\ue^{\ui\vec k\cdot\vec r'}\paren{-\paren{1+\ui\xi}V(r')-\frac{c}{r'}V_r(r')\vec\sigma\cdot\paren{\vec r'\times\vec k'}}\ue^{-\ui\vec k'\cdot\vec r'}\chi_\sigma\\
  =&\frac{1}{4\pi}\frac{2m}{\hbar^2}\int\ud^3r'\ue^{\ui\paren{\vec k-\vec k'}\cdot\vec r'}\paren{1+\ui\xi}V(r')\chi_\sigma+\frac{1}{4\pi}\frac{2m}{\hbar^2}\int\ud^3r'\ue^{\ui\paren{\vec k-\vec k'}\cdot\vec r'}\frac{c}{r'}V_r(r')\vec\sigma\cdot\paren{\vec r'\times\vec k'}\chi_\sigma
  \intertext{Let $\vec q=\vec k-\vec k'$, $q=2k\sin\dfrac{\theta}2$}
  f_\sigma=&\frac{1}{4\pi}\frac{2m}{\hbar^2}\int\ud\Omega\ud r'r'^2\ue^{\ui qr'\cos\theta'}\paren{1+\ui\xi}V(r')\chi_\sigma\\
  &+\frac{1}{4\pi}\frac{2m}{\hbar^2}\int\ud\Omega\ud r'r'^2\ue^{\ui qr'\cos\theta'}cV_r(r')\vec\sigma\cdot\paren{\hat r'\times\vec k'}\chi_\sigma
  \intertext{Due to the rotational symmetry around $\vec q$}
  f_\sigma=&\frac{1}{4\pi}\frac{2m}{\hbar^2}\int\ud\Omega\ud r'r'^2\ue^{\ui qr'\cos\theta'}\paren{1+\ui\xi}V(r')\chi_\sigma\\
  &+\frac{1}{4\pi}\frac{2m}{\hbar^2}\int\ud\Omega\ud r'r'^2\cos\theta'\ue^{\ui qr'\cos\theta'}cV_r(r')\vec\sigma\cdot\paren{\hat q\times\vec k'}\chi_\sigma\\
  =&\frac{m}{\hbar^2}\int\ud r'r'^2\paren{1+\ui\xi}V(r')\chi_\sigma\int_{-1}^1\ud\cos\theta'\ue^{\ui qr'\cos\theta'}\\
  &+\frac{m}{\hbar^2}\int\ud r'r'^2cV_r(r')\vec\sigma\cdot\paren{\hat q\times\vec k'}\chi_\sigma\int_{-1}^1\ud\cos\theta'\cos\theta'\ue^{\ui qr'\cos\theta'}\\
  =&\frac{m}{\hbar^2}\int\ud r'r'^2\paren{1+\ui\xi}V(r')\chi_\sigma
  \frac{\ue^{\ui qr'}-\ue^{-\ui qr'}}{\ui qr'}\\
  &+\frac{m}{\hbar^2}\int\ud r'r'^2cV_r(r')\vec\sigma\cdot\paren{\hat q\times\vec k'}\chi_\sigma
  \paren{
    \ue^{\ui qr'}\paren{\frac{1}{\ui qr'}+\frac{1}{q^2r'^2}}
    +\ue^{-\ui qr'}\paren{\frac{1}{\ui qr'}-\frac{1}{q^2r'^2}}
  }\\
  =&\frac{2m}{\hbar^2q}\int\ud r'r'\paren{1+\ui\xi}V(r')\sin{qr'}\chi_\sigma\\
  &+\frac{2m\ui}{\hbar^2q^2}\int\ud r'r'cV_r(r')\vec\sigma\cdot\paren{\vec q\times\vec k'}\chi_\sigma\paren{\frac{\sin{qr'}}{qr'}-\cos{qr'}}
  \intertext{Substitute in $\hat n$}
  =&\frac{2m}{\hbar^2q}\int\ud r'r'\paren{1+\ui\xi}V(r')\sin{qr'}\chi_\sigma\\
  &+\vec\sigma\cdot\hat n\frac{2\ui mk^2c}{\hbar^2q^2}\sin\theta\int\ud r'r'V_r(r')\paren{\frac{\sin{qr'}}{qr'}-\cos{qr'}}\chi_\sigma
  \intertext{Therefore,}
  A\paren{\theta}=&\frac{2m}{\hbar^2q}\int\ud r'r'\paren{1+\ui\xi}V(r')\sin{qr'}\\
  B\paren{\theta}=&\frac{2\ui mk^2c}{\hbar^2q^2}\sin\theta\int\ud r'r'V_r(r')\paren{\frac{\sin{qr'}}{qr'}-\cos{qr'}}
}
\subsection{}
For $V$ being a step function,
\eqar{
  A\paren{\theta}=&\frac{2mV_0\paren{1+\ui\xi}}{\hbar^2q}\int_0^R\ud r'r'\sin{qr'}\\
  =&\frac{2mV_0\paren{1+\ui\xi}}{\hbar^2q^3}\paren{\sin{qR}-qR\cos{qR}}\\
  B\paren{\theta}=&-\frac{2\ui mk^2c}{\hbar^2q^2}\sin\theta\int\ud r'r'V_0\delta\paren{r'-R}\paren{\frac{\sin{qr'}}{qr'}-\cos{qr'}}\\
  =&-\frac{2\ui mV_0k^2c}{\hbar^2q^3}\sin\theta \paren{\sin{qR}-qR\cos{qR}}
}
Average over input and sum over output to get total scattering cross section
\eqar{
  \diff{\sigma}{\Omega}=&\frac12\Tr\paren{\paren{A^*+\vec\sigma^*\cdot\hat nB^*}\paren{A+\vec\sigma\cdot\hat nB}}
  \intertext{Since $\Tr\paren{\sigma_n}=0$}
  \diff{\sigma}{\Omega}=&\frac12\paren{\abs{A}^2+\abs{B}^2}\\
  =&2\abs{\frac{mV_0\paren{1+\ui\xi}}{\hbar^2q^3}\paren{\sin{qR}-qR\cos{qR}}}^2
  +2\abs{\frac{mV_0k^2c}{\hbar^2q^3}\sin\theta \paren{\sin{qR}-qR\cos{qR}}}^2\\
  =&2\paren{\frac{mV_0}{\hbar^2q^3}\paren{\sin{qR}-qR\cos{qR}}}^2\paren{1+\xi^2+k^4c^2\sin^2\theta}
}
\section{}
\subsection{}
Time dependent Schr\"odinger equation
\eqar{
  \ui\diff{}{t}|\psi\rangle=&H|\psi\rangle
  \intertext{Expansion,}
  |\psi\rangle=&\sum_nc_n\ue^{\ui\theta_n}|\psi_n\rangle
  \intertext{where}
  H|\psi_n\rangle=&E_n|\psi_n\rangle\\
  0=&\ui\diff{}{t}\sum_nc_n\ue^{\ui\theta_n}|\psi_n\rangle-H\sum_nc_n\ue^{\ui\theta_n}|\psi_n\rangle\\
  0=&\ui\sum_nc_n\diff{\ue^{\ui\theta_n}}{t}|\psi_n\rangle+\ui\sum_n\diff{c_n}{t}\ue^{\ui\theta_n}|\psi_n\rangle+\ui\sum_nc_n\ue^{\ui\theta_n}\diff{}{t}|\psi_n\rangle-\sum_nc_nE_n\ue^{\ui\theta_n}|\psi_n\rangle\\
  0=&\langle\psi_l|\sum_n\diff{c_n}{t}\ue^{\ui\theta_n}|\psi_n\rangle+\langle\psi_l|\sum_nc_n\ue^{\ui\theta_n}\diff{}{t}|\psi_n\rangle\\
  \diff{c_l}{t}=&-\sum_nc_n\ue^{\ui\paren{\theta_n-\theta_l}}\langle\psi_l|\diff{}{t}\psi_n\rangle
}
\subsection{}
\eqar{
  \diff{c_l^{(1)}}{t}=&-\sum_nc^{(0)}_n\ue^{\ui\paren{\theta_n-\theta_l}}\langle\psi_l|\diff{}{t}\psi_n\rangle\\
  =&-\sum_n\delta_{nm}\ue^{\ui\gamma_mt}\ue^{\ui\paren{\theta_n-\theta_l}}\langle\psi_l|\diff{}{t}\psi_n\rangle\\
  =&-\ue^{\ui\gamma_mt}\ue^{\ui\paren{\theta_m-\theta_l}}\langle\psi_l|\diff{}{t}\psi_m\rangle\\
  c_l^{(1)}(t)=&c_l^{(1)}(0)-\int_0^t\ud t'\ue^{\ui\gamma_mt'}\ue^{\ui\paren{\theta_m(t')-\theta_l(t')}}\langle\psi_l(t')|\diff{}{t'}\psi_m(t')\rangle
}
\subsection{}
\end{document}
