\documentclass[10pt,fleqn]{article}
% \usepackage[journal=rsc]{chemstyle}
% \usepackage{mhchem}
\usepackage{amsmath}
\usepackage{amssymb}
\usepackage{amsfonts}
\usepackage{esint}
\usepackage{bbm}
\usepackage{amscd}
\usepackage{picinpar}
\usepackage[pdftex]{graphicx}
\usepackage{indentfirst}
\usepackage{wrapfig}
\usepackage{units}
\usepackage{textcomp}
\usepackage[utf8x]{inputenc}
\usepackage{feyn}
\usepackage{feynmp}
\DeclareGraphicsRule{*}{mps}{*}{}
\newcommand{\ud}{\mathrm{d}}
\newcommand{\ue}{\mathrm{e}}
\newcommand{\ui}{\mathrm{i}}
\newcommand{\res}{\mathrm{Res}}
\newcommand{\Tr}{\mathrm{Tr}}
\newcommand{\dsum}{\displaystyle\sum}
\newcommand{\dprod}{\displaystyle\prod}
\newcommand{\dlim}{\displaystyle\lim}
\newcommand{\dint}{\displaystyle\int}
\newcommand{\fsno}[1]{{\!\not\!{#1}}}
\newcommand{\eqar}[1]
{
  \begin{align*}
    #1
  \end{align*}
}
\newcommand{\texp}[2]{\ensuremath{{#1}\times10^{#2}}}
\newcommand{\dexp}[2]{\ensuremath{{#1}\cdot10^{#2}}}
\newcommand{\eval}[2]{{\left.{#1}\right|_{#2}}}
\newcommand{\paren}[1]{{\left({#1}\right)}}
\newcommand{\lparen}[1]{{\left({#1}\right.}}
\newcommand{\rparen}[1]{{\left.{#1}\right)}}
\newcommand{\abs}[1]{{\left|{#1}\right|}}
\newcommand{\sqr}[1]{{\left[{#1}\right]}}
\newcommand{\crly}[1]{{\left\{{#1}\right\}}}
\newcommand{\angl}[1]{{\left\langle{#1}\right\rangle}}
\newcommand{\tpdiff}[4][{}]{{\paren{\frac{\partial^{#1} {#2}}{\partial {#3}{}^{#1}}}_{#4}}}
\newcommand{\tpsdiff}[4][{}]{{\paren{\frac{\partial^{#1}}{\partial {#3}{}^{#1}}{#2}}_{#4}}}
\newcommand{\pdiff}[3][{}]{{\frac{\partial^{#1} {#2}}{\partial {#3}{}^{#1}}}}
\newcommand{\diff}[3][{}]{{\frac{\ud^{#1} {#2}}{\ud {#3}{}^{#1}}}}
\newcommand{\psdiff}[3][{}]{{\frac{\partial^{#1}}{\partial {#3}{}^{#1}} {#2}}}
\newcommand{\sdiff}[3][{}]{{\frac{\ud^{#1}}{\ud {#3}{}^{#1}} {#2}}}
\newcommand{\tpddiff}[4][{}]{{\left(\dfrac{\partial^{#1} {#2}}{\partial {#3}{}^{#1}}\right)_{#4}}}
\newcommand{\tpsddiff}[4][{}]{{\paren{\dfrac{\partial^{#1}}{\partial {#3}{}^{#1}}{#2}}_{#4}}}
\newcommand{\pddiff}[3][{}]{{\dfrac{\partial^{#1} {#2}}{\partial {#3}{}^{#1}}}}
\newcommand{\ddiff}[3][{}]{{\dfrac{\ud^{#1} {#2}}{\ud {#3}{}^{#1}}}}
\newcommand{\psddiff}[3][{}]{{\frac{\partial^{#1}}{\partial{}^{#1} {#3}} {#2}}}
\newcommand{\sddiff}[3][{}]{{\frac{\ud^{#1}}{\ud {#3}{}^{#1}} {#2}}}
\usepackage{fancyhdr}
\usepackage{multirow}
\usepackage{fontenc}
% \usepackage{tipa}
\usepackage{ulem}
\usepackage{color}
\usepackage{cancel}
\newcommand{\hcancel}[2][black]{\setbox0=\hbox{#2}%
  \rlap{\raisebox{.45\ht0}{\textcolor{#1}{\rule{\wd0}{1pt}}}}#2}
\pagestyle{fancy}
\setlength{\headheight}{67pt}
\fancyhead{}
\fancyfoot{}
\fancyfoot[C]{\thepage}
\fancyhead[R]
{
  Yichao Yu\\
  Physics 251b Final\\
}
\renewcommand{\footruleskip}{0pt}
\renewcommand{\headrulewidth}{0.4pt}
\renewcommand{\footrulewidth}{0pt}
\addtolength{\hoffset}{-1.3cm}
\addtolength{\voffset}{-2cm}
\addtolength{\textwidth}{3cm}
\addtolength{\textheight}{2.5cm}
\renewcommand{\footskip}{10pt}
\setlength{\headwidth}{\textwidth}
\setlength{\headsep}{20pt}
\setlength{\marginparwidth}{0pt}
\parindent=0pt
\renewcommand{\thesection}{\arabic{section}.}
\renewcommand{\thesubsection}{(\alph{subsection})}
\renewcommand{\thesubsubsection}{\roman{subsubsection}.}
\begin{document}
\section{}
\subsection{}
Communator of each component
\eqar{
  \sqr{L_i+g_0S_i,J_j}=&\sqr{L_i+g_0S_i,L_j+S_j}\\
  =&\sqr{L_i,L_j}+g_0\sqr{S_i,S_j}\\
  =&\ui\hbar\varepsilon_{ijk}\paren{L_k+g_0S_k}\\
  \sqr{L_i+g_0S_i,\hat n\cdot\vec J}=&\ui\hbar\varepsilon_{ijk}n_j\paren{L_k+g_0S_k}\\
  =&\ui\hbar \paren{\hat n\times\paren{\vec L+g_0\vec S}}_i\\
  \sqr{\vec L+g_0\vec S,\hat n\cdot\vec J}=&\ui\hbar\varepsilon_{ijk}n_j\paren{L_k+g_0S_k}\\
  =&\ui\hbar\hat n\times\paren{\vec L+g_0\vec S}
}
Therefore for any $\vec n$
\eqar{
  &\ui\hbar\hat n\times\langle 0|\vec L+g_0\vec S|0\rangle\\
  =&\langle 0|\sqr{L_i+g_0S_i,\hat n\cdot\vec J}|0\rangle\\
  =&\langle 0|\sqr{L_i+g_0S_i,0}|0\rangle\\
  =&0\\
  &\langle 0|\vec L+g_0\vec S|0\rangle\\
  =&0
}
This is a special case of the Wigner-Eckart Theorem because the $|0\rangle$ state
is spherical symmetric. The physical origin of the factor $g_0$ is the low energy
limit of the Dirac equation of electron (and QED corrections on top of it).
\subsection{}
\section{}
\section{}
\subsection{}
Radial component of $\vec j$
\eqar{
  j_r=&\frac{\hbar}{2m\ui}\paren{\psi^*\pdiff{}{r}\psi-\psi\pdiff{}{r}\psi^*}\\
  =&\frac{\hbar}{m}\Im\paren{\psi^*\pdiff{}{r}\psi}
}
The part terms that is due to interference (for $\psi=\psi_1+\psi_2$)
\eqar{
  j_r'=&j_r-j_{r1}-j_{r2}\\
  =&\frac{\hbar}{m}\Im\paren{\psi^*\pdiff{}{r}\psi}-\frac{\hbar}{m}\Im\paren{\psi_1^*\pdiff{}{r}\psi_1}-\frac{\hbar}{m}\Im\paren{\psi_2^*\pdiff{}{r}\psi_2}\\
  =&\frac{\hbar}{m}\Im\paren{\psi_1^*\pdiff{}{r}\psi_2}+\frac{\hbar}{m}\Im\paren{\psi_2^*\pdiff{}{r}\psi_1}
}
Scattering wave function
\eqar{
  \psi=&\ue^{\ui kr\cos\theta}+f\frac{\ue^{\ui kr}}{r}
  \intertext{current density}
  j_r'=&\frac{\hbar}{m}\Im\paren{f\ue^{-\ui kr\cos\theta}\pdiff{}{r}\frac{\ue^{\ui kr}}{r}+f^*\frac{\ue^{-\ui kr}}{r}\pdiff{}{r}\ue^{\ui kr\cos\theta}}\\
  =&\frac{\hbar}{m}\Im\paren{f\ue^{-\ui kr\cos\theta}\frac{r\ui k-1}{r^2}\ue^{\ui kr}+f^*\ui k\cos\theta\frac{\ue^{-\ui kr}}{r}\ue^{\ui kr\cos\theta}}
  \intertext{Ignoring $r^{-2}$ term for large $r$}
  j_r'\approx&\frac{\hbar k}{m}\frac{1}{r}\Im\paren{\ui f\ue^{-\ui kr\cos\theta}\ue^{\ui kr}+\ui f^*\cos\theta\ue^{-\ui kr}\ue^{\ui kr\cos\theta}}\\
  =&\frac{\hbar k}{m}\frac{1}{r}\Im\paren{\ui \ue^{\ui kr\paren{\cos\theta-1}}f^*\cos\theta+\ui \ue^{\ui kr\paren{1-\cos\theta}}f}
}
\subsection{}
\subsection{}
\subsection{}
\section{}
\subsection{}
\subsection{}
\section{}
\subsection{}
\subsection{}
\subsection{}
\end{document}
